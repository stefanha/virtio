316 & 05 Mar 2014 & Michael S. Tsirkin & { legacy framing: scsi host
 } \\
\hline
315 & 05 Mar 2014 & Michael S. Tsirkin & { legacy message framing: console device
 } \\
\hline
314 & 05 Mar 2014 & Michael S. Tsirkin & { block: legacy message framing
 } \\
\hline
313 & 05 Mar 2014 & Michael S. Tsirkin & { message framing: rusty's comments

generic note on message framing

specific requirements listed for net device only
 } \\
\hline
312 & 05 Mar 2014 & Michael S. Tsirkin & { legacy devices: get rid of MUST assume

as Rusty points out MUST assume is not very good requirement.

clarify it.
 } \\
\hline
311 & 05 Mar 2014 & Michael S. Tsirkin & { transitional driver features: fix typos noted by Rusty
 } \\
\hline
310 & 03 Mar 2014 & Rusty Russell & { Formatting: use latex-style quoting everywhere.

Doesn't look any different, but consistent.
 } \\
\hline
309 & 03 Mar 2014 & Rusty Russell & { Use ellipsis (aka \textbackslash ldots) everywhere.

And use the ellipsis package, which makes it symmetrical.
 } \\
\hline
308 & 03 Mar 2014 & Rusty Russell & { PCI: Tighten requirements.

1) make it clear that queue_enable is 0 on reset.

2) device MUST present a  VIRTIO_PCI_CAP_DEVICE_CFG if needed for type.
 } \\
\hline
307 & 02 Mar 2014 & Michael S. Tsirkin & { initialization: minor clarification

Change accepted on Virtio TC Meeting Minutes: Feb 25, 2014

"it" could refer to failed bit or the driver.

clarify.
 } \\
\hline
306 & 02 Mar 2014 & Michael S. Tsirkin & { fix rfc2119 reference

Change accepted on Virtio TC Meeting Minutes: Feb 25, 2014

VIRTIO-68

Cc: Patrick Durusau <patrick@durusau.net>
 } \\
\hline
305 & 02 Mar 2014 & Michael S. Tsirkin & { VIRTIO-67: fix html redirects

Change accepted on Virtio TC Meeting Minutes: Feb 25, 2014

1. oasis switched to https

2. Red Hat  is www.redhat.com

Cc: Patrick Durusau <patrick@durusau.net>
 } \\
\hline
304 & 02 Mar 2014 & Michael S. Tsirkin & { feedback: clarify device status bits

Change accepted on Virtio TC Meeting Minutes: Feb 25, 2014

VIRTIO-70

Cc: Patrick Durusau <patrick@durusau.net>
 } \\
\hline
303 & 02 Mar 2014 & Michael S. Tsirkin & { legacy interface: move to terminology

Change accepted on Virtio TC Meeting Minutes: Feb 25, 2014

VIRTIO-64

Cc: Patrick Durusau <patrick@durusau.net
 } \\
\hline
302 & 02 Mar 2014 & Michael S. Tsirkin & { introduction: add link to 0.9.5 specification

Change accepted on Virtio TC Meeting Minutes: Feb 25, 2014

this version replaces it, so it's a non normative reference.

VIRTIO-69

note: the link is added here but isn't used yet: will be used

when we cleanup terminology definitions, by

addressing VIRTIO-64

Cc: Patrick Durusau <patrick@durusau.net
 } \\
\hline
301 & 02 Mar 2014 & Michael S. Tsirkin & { non-transitional devices with legacy drivers

Change accepted on Virtio TC Meeting Minutes: Feb 25, 2014

weaken hacky requirements helpful for graceful failure

for non transitional PCI devices from MUST to SHOULD.

It's nice to have but it's not like it makes things work, and you

can avoid trouble simply by using the most recent drivers.

also move them out to a separate section
 } \\
\hline
300 & 02 Mar 2014 & Michael S. Tsirkin & { conformance: document two types of devices

Change accepted on Virtio TC Meeting Minutes: Feb 25, 2014

document that there are two conformance levels
 } \\
\hline
299 & 02 Mar 2014 & Michael S. Tsirkin & { legacy device initialization: confirmance statements

Change accepted on Virtio TC Meeting Minutes: Feb 25, 2014
 } \\
\hline
298 & 02 Mar 2014 & Michael S. Tsirkin & { legacy virtqueue layout: confirmance

Change accepted on Virtio TC Meeting Minutes: Feb 25, 2014
 } \\
\hline
297 & 02 Mar 2014 & Michael S. Tsirkin & { legacy: make all notes on endian-ness confirmance clauses

Change accepted on Virtio TC Meeting Minutes: Feb 25, 2014
 } \\
\hline
296 & 02 Mar 2014 & Michael S. Tsirkin & { legacy feature bits: confirmance statements

Change accepted on Virtio TC Meeting Minutes: Feb 25, 2014
 } \\
\hline
295 & 02 Mar 2014 & Michael S. Tsirkin & { leacy: layout detection confirmance

Change accepted on Virtio TC Meeting Minutes: Feb 25, 2014
 } \\
\hline
294 & 02 Mar 2014 & Michael S. Tsirkin & { legacy pci layout: extra confirmance statement

Change accepted on Virtio TC Meeting Minutes: Feb 25, 2014
 } \\
\hline
293 & 02 Mar 2014 & Michael S. Tsirkin & { legacy pci layout: confirmance statements

Change accepted on Virtio TC Meeting Minutes: Feb 25, 2014
 } \\
\hline
292 & 02 Mar 2014 & Michael S. Tsirkin & { legacy: make message framing normative

Change accepted on Virtio TC Meeting Minutes: Feb 25, 2014

TODO: we really should be more specific
 } \\
\hline
291 & 02 Mar 2014 & Michael S. Tsirkin & { legacy: make note on legacy VQ endian-ness normative

Change accepted on Virtio TC Meeting Minutes: Feb 25, 2014
 } \\
\hline
290 & 02 Mar 2014 & Michael S. Tsirkin & { Legacy Interface: Device Configuration Space

Change accepted on Virtio TC Meeting Minutes: Feb 25, 2014

legacy has no generation field.

add SHOULD statement to document multi-byte field

access rules.
 } \\
\hline
289 & 02 Mar 2014 & Michael S. Tsirkin & { legacy: clarify general note on endian-ness

Change accepted on Virtio TC Meeting Minutes: Feb 25, 2014

this is a non normative section.

we merely mention that details are given

for each device.
 } \\
\hline
288 & 02 Mar 2014 & Michael S. Tsirkin & { content: explain that legacy support is optional

Change accepted on Virtio TC Meeting Minutes: Feb 25, 2014
 } \\
\hline
287 & 02 Mar 2014 & Michael S. Tsirkin & { drop /* LEGACY version was not little endian */

Two issues with the comment:

	- it mixes legacy documentation in main part of the spec

	- it says what format *isn't* - instead of what it *is*

Now that we have documented that LE can mean

legacy endian, there's no need for the comment.

Resolves issues:

	VIRTIO-58

Change accepted on Virtio TC Meeting Minutes: Feb 25, 2014
 } \\
\hline
286 & 27 Feb 2014 & Rusty Russell & { Fixed path of linux version of virtio_ring.h
 } \\
\hline
285 & 26 Feb 2014 & Pawel Moll & { mmio: Clarify normative requirement on QueueNum

Minor fix: add explicit reference to QueueNumMax in the

normative paragraph describing QueueNum.
 } \\
\hline
284 & 26 Feb 2014 & Pawel Moll & { mmio: Fix double register macro
Minor fix: commit 238 wrapped register names in \textbackslash field\{\}s

and modified one of the register table macros, but missed

the other one.
 } \\
\hline
283 & 26 Feb 2014 & Pawel Moll & { mmio: Fix Device Tree example

Minor fix: the size of 0x100 was obviously wrong,

as it didn't allow for configuration space.
 } \\
\hline
282 & 26 Feb 2014 & Cornelia Huck & { introduction: typo in terminology section

s/device/driver/ for the transitional driver description.
 } \\
\hline
281 & 26 Feb 2014 & Rusty Russell & { ccw: Fix requirements for processing adapter interrupts.

We currently mandate that the driver clears the summary indicator

before processing the queue indicator; this is bogus, as the requirement

for interrupt avoidance is rather that the driver unsets the summary

indicator before before it stops looking at the queue indicator.

In fact, the best way to get a race-free implementation of the interrupt

handler is to process the queue indicators twice; let's add a recommondation

to do that.
 } \\
\hline
280 & 26 Feb 2014 & Rusty Russell & { VIRTIO-45: Add a reserved ID for Timer/Clock device

Just add a reserved ID for Timer/Clock device. There is no work

on it yet but it is nice to have the ID which could be used safely

in preliminary implementations.
 } \\
\hline
279 & 26 Feb 2014 & Rusty Russell & { VIRTIO-28: Deprecate balloon device, add number for new one.
 } \\
\hline
278 & 26 Feb 2014 & Rusty Russell & { Feedback: VIRTIO-77 Conformance clause.

Now we have grouped all the normative statements, the conformance

clauses for drivers and devices can simply reference them.
 } \\
\hline
277 & 26 Feb 2014 & Rusty Russell & { Feedback: Separate normative requirements for Reserved Feature Bits.
 } \\
\hline
276 & 26 Feb 2014 & Rusty Russell & { Feedback: SCSI: Separate normative and descriptive texts.

This could use some more rigour, I think: there are still many

implied requirements which could be called out.
 } \\
\hline
275 & 26 Feb 2014 & Rusty Russell & { Feedback: console \& entropy: separate normative and descriptive texts.
 } \\
\hline
274 & 26 Feb 2014 & Rusty Russell & { Feedback: block: separate normative and descriptive text.
 } \\
\hline
273 & 26 Feb 2014 & Rusty Russell & { Feedback: net: separate normative and instructional text.
 } \\
\hline
272 & 26 Feb 2014 & Rusty Russell & { Feedback: CCW: Separate normative and descriptive sections.
 } \\
\hline
271 & 26 Feb 2014 & Rusty Russell & { Feedback: MMIO: Separate normative and descriptive text.

The section on initialization is now non-normative.
 } \\
\hline
270 & 26 Feb 2014 & Rusty Russell & { Feedback: PCI: Separate explanatory and normative text.

Rather than treat selectors 0 and 1 as special, the wording for features

is made more general (though still the same effect).

I split the interrupt handler into a separate subsection: it was

misleading because it didn't handle configuration interrupts until

the next section.  It's also non-normative.
 } \\
\hline
269 & 26 Feb 2014 & Rusty Russell & { Feedback: Separate the rest of chapter 2 into normative vs explanatory.

The big change here is in introducing new subsections for interrupt and notification

suppression, and moving all requirements into them.

The example processing loop is also moved into a note, to show clearly

that it's not normative.
 } \\
\hline
268 & 26 Feb 2014 & Rusty Russell & { Feedback: Normative split for Basic Facilities of a Virtio Device / Virtqueues / Message Framing
 } \\
\hline
267 & 26 Feb 2014 & Rusty Russell & { Feedback: Normative split in Basic Facilities of a Virtio Device / Virtqueues
 } \\
\hline
266 & 26 Feb 2014 & Rusty Russell & { Feedback: split Basic Facilities feature bits and config space into normative.

Split text into descriptive and normative.
 } \\
\hline
265 & 26 Feb 2014 & Rusty Russell & { Feedback: add normative marker.
From \url{http://docs.oasis-open.org/templates/TCHandbook/ConformanceGuidelines.html:}

   Normative statements MUST be referenceable so that a statement may be

   referenced from another part of a specification, but more importantly

   so they can be referenced from Conformance Clauses.
 } \\
\hline
264 & 26 Feb 2014 & Rusty Russell & { Feedback: 2.1 Device Status field: Separate description from normative.

Start with explanation, progress to normative requirements.
 } \\
\hline
263 & 26 Feb 2014 & Rusty Russell & { Feedback: move legacy/transitional definitions into terminology.
 } \\
\hline
262 & 26 Feb 2014 & Rusty Russell & { Feedback: hoist the one legacy-related requirement out of legacy section.

This requirement applies to any system which *did* have legacy drivers.
 } \\
\hline
261 & 26 Feb 2014 & Rusty Russell & { Feedback: add old draft to normative references (VIRTIO-77)
 } \\
\hline
260 & 26 Feb 2014 & Rusty Russell & { Feedback: use proper list in introduction (VIRTIO-82)

Also avoid extra spacing before footnote markers.
 } \\
\hline
259 & 26 Feb 2014 & Rusty Russell & { Feedback: move new device design section to Appendix.

It's non-normative.
 } \\
\hline
258 & 26 Feb 2014 & Rusty Russell & { Feedback: Bug TAB-553 (VIRTIO-76)

Haven't marked them non-normative yet, but it makes sense to put the header

in an appendix.
 } \\
\hline
257 & 26 Feb 2014 & Rusty Russell & { Feedback: TAB-555 Bad sub-sectioning (VIRTIO-80)
 } \\
\hline
256 & 26 Feb 2014 & Rusty Russell & { Feedback: TAB-557 Spelling errors, etc (VIRTIO-75)
 } \\
\hline
255 & 26 Feb 2014 & Rusty Russell & { PCI: better document driver and device requirements

Feedback:

10) 4.1.3.1.2 Queue Vector Configuration

Some of the information from section 8.4 needs to be moved to

here, for example that the device may have an MSI-X table size

other than 2048.

Otherwise, this reads as though the MSI-X table must always have

2048 entries.

11) Please explicitly describe the device behavior when writing

a vector value beyond the MSI-X table size.

Address these comments.

Cc: Arun Subbarao <asubbarao@lnxw.com>
 } \\
\hline
254 & 26 Feb 2014 & Rusty Russell & { feedback: minor wording cleanups

We already mention requirement for natural width

accesses for non device specific configuration.

Don't repeat this in legacy section.

Further, mention virtio pci structure in

preamble to help link sections together.

Cc: Arun Subbarao <asubbarao@lnxw.com>

Conflicts:

	content.tex
 } \\
\hline
253 & 26 Feb 2014 & Rusty Russell & { SCSI: fix up more fields.
Some missing \textbackslash field\{\} markings, and a few redundant "the XXX field".
 } \\
\hline
252 & 20 Feb 2014 & Rusty Russell & { SCSI: missing space.
 } \\
\hline
251 & 19 Feb 2014 & Rusty Russell & { Gratuitous Packet Sending: clarify wording.

The device can ask, not the driver.
 } \\
\hline
250 & 19 Feb 2014 & Rusty Russell & { net: fix incorrect reference.

It pointed into the block section for some reason.
 } \\
\hline
249 & 13 Feb 2014 & Cornelia Huck & { ccw: padding annotations

Remove __packed__ annotation from all ccw structures that don't need it,

and make the length requirements explicit for those that do.

This is the part of the patch to resolve VIRTIO-56 that had been missed.
 } \\
\hline
248 & 12 Feb 2014 & Michael S. Tsirkin & { PCI: minor wording change

Since access width requirement is a confirmance clause, make it

explicit that it applies to 4,2 and 1 byte fields.

Also explain what happens to fields of other widths (such as

the 6 byte MAC).
 } \\
\hline
247 & 12 Feb 2014 & Michael S. Tsirkin & { content: more strict confirmance language

Correct new language to explicitly use MAY/SHOULD/MUST

in more places or simply drop the somewhat vague "can" where

we are describing the only way to operate the device.

Most of the changes are in the PCI section.
 } \\
\hline
246 & 12 Feb 2014 & Michael S. Tsirkin & { introduction: address lnovich comments

generally list of buses is out of date, list all supported buses.

Drop explicit mention of lguest since it's not part of the spec.
 } \\
\hline
245 & 12 Feb 2014 & Michael S. Tsirkin & { abstract: address lnovich comment

lnovich@redhat.com suggested rewording abstract,

making the following point:

. from what point of view is virtio like a physical device?

  it's very different from host POV

. "the guest" appears out of nowhere. It's the guest that runs

  in the vm of course.

. "not all that different" means similar so there's not need to

  be verbose

Address this comment
 } \\
\hline
244 & 12 Feb 2014 & Rusty Russell & { VIRTIO-55: Add a reserved ID for GPU devices

As existing work on virtio-gpu is using device ID 16, reflect this in

the spec. This closes out VIRTIO-55.

As per minutes:
        \url{https://lists.oasis-open.org/archives/virtio/201402/msg00121.html}
 } \\
\hline
243 & 12 Feb 2014 & Rusty Russell & { Fix S390 normative references.

As pointed out in TAB-539 and TAB-540:

- Add an URL to the documents. (Unfortunately, there is no link that

  always points to the latest version.)

- State that we include any future revisions as well.

As per minutes:
        \url{https://lists.oasis-open.org/archives/virtio/201402/msg00121.html}
 } \\
\hline
242 & 12 Feb 2014 & Rusty Russell & { ccw: Further use of RFC2119 language.

Some more instances of MAY and SHOULD, as reported in TAB comments

TAB-548 and TAB-550.

As per minutes:
        \url{https://lists.oasis-open.org/archives/virtio/201402/msg00121.html}
 } \\
\hline
241 & 12 Feb 2014 & Rusty Russell & { PCI: explicitly document ISR status field

Feedback on ISR status register:

	It would be helpful if this section provided the meaning of each

	bit in the register.

ISR use is scattered all around the place.

Add a section describing the format and semantics.

[ Merged to combine with new ISR-specific section --RR ]

As per minutes:
        \url{https://lists.oasis-open.org/archives/virtio/201402/msg00121.html}

Cc: Arun Subbarao <asubbarao@lnxw.com>
 } \\
\hline
240 & 12 Feb 2014 & Rusty Russell & { PCI: consistent device/PCI configuration space

Re section:

4.1.3.4 Notification of Device Configuration Changes

Feedback:

	Please use "PCI configuration space" and "device configuration

	state" consistently, without abbreviation. For example, from the

	first sentence it looks like "device configuration state" can be

	changed, but the first bullet claims it's "configuration space".

	So, which one? Does "configuration space" mean "PCI configuration

	space" or is it a synonym for "device configuration state"?

	Because those are two different things; the driver needs to know

	what exactly to rescan.

As per minutes:
        \url{https://lists.oasis-open.org/archives/virtio/201402/msg00121.html}

Cc: Arun Subbarao <asubbarao@lnxw.com>
 } \\
\hline
239 & 12 Feb 2014 & Rusty Russell & { Feedback \#8: Applied.

[ Includes fixup! removing MSI-X ]

As per minutes:
        \url{https://lists.oasis-open.org/archives/virtio/201402/msg00121.html}
 } \\
\hline
238 & 12 Feb 2014 & Rusty Russell & { Feedback \#7: Applied

Some minor merging required.

As per minutes:
    \url{https://lists.oasis-open.org/archives/virtio/201402/msg00121.html}
 } \\
\hline
237 & 12 Feb 2014 & Rusty Russell & { Feedback \#6: Applied

As per minutes:
	\url{https://lists.oasis-open.org/archives/virtio/201402/msg00121.html}
 } \\
\hline
236 & 12 Feb 2014 & Rusty Russell & { Feedback \#5: Applied.

As per minutes:
	\url{https://lists.oasis-open.org/archives/virtio/201402/msg00121.html}
 } \\
\hline
235 & 12 Feb 2014 & Rusty Russell & { Feedback \#4: applied.

As per minutes:
	\url{https://lists.oasis-open.org/archives/virtio/201402/msg00121.html}
 } \\
\hline
234 & 12 Feb 2014 & Rusty Russell & { PCI: minor changes for previous patch.
 } \\
\hline
233 & 12 Feb 2014 & Rusty Russell & { PCI: rearrange it all

This is the re-arrangement originally suggested by Rusty,

except I made some fixes and also tweaked a couple of places

where behaviour changes where suggested - if we want these,

they should go in separately.

Rearrange discovery section to make it clearer what goes on.

Wording changes MUST/MAY/etc.  Clarify cfg gateway use.  No

behavioural changes.

[ Merged "fixup! PCI: rearrange it all" --RR ]

As per minutes:
        \url{https://lists.oasis-open.org/archives/virtio/201402/msg00121.html}
 } \\
\hline
232 & 12 Feb 2014 & Rusty Russell & { PCI: rearrange it all

This is the re-arrangement originally suggested by Rusty,

except I made some fixes and also tweaked a couple of places

where behaviour changes where suggested - if we want these,

they should go in separately.

Rearrange discovery section to make it clearer what goes on.

Wording changes MUST/MAY/etc.  Clarify cfg gateway use.  No

behavioural changes.
 } \\
\hline
231 & 12 Feb 2014 & Rusty Russell & { C struct specifications.

Explicitly specify that our C struct specifications are without padding,

and add some definitions for our integer data types.

[ Rusty - added /* comments */ and removed redundant old le* explanation ]
 } \\
\hline
225 & 10 Feb 2014 & Rusty Russell & { REVERT LAST 15 JUNK COMMITS.

Back to r211.  It's been a long day.
 } \\
\hline
224 & 10 Feb 2014 & Rusty Russell & { patch feedback-8-9.patch
 } \\
\hline
223 & 10 Feb 2014 & Rusty Russell & { patch feedback-8-7.patch
 } \\
\hline
222 & 10 Feb 2014 & Rusty Russell & { patch feedback-8-6.patch
 } \\
\hline
221 & 10 Feb 2014 & Rusty Russell & { patch feedback-8-5.patch
 } \\
\hline
220 & 10 Feb 2014 & Rusty Russell & { feedback: s/virtio header/virtio common configuration/

While most places now sat virtio common configuration

structure, some places still use the term virtio header.

Since it's not necessarily before the

common configuration anymore, rename it

to virtio common configuration structure for consistency.

Cc: Arun Subbarao <asubbarao@lnxw.com>
 } \\
\hline
219 & 10 Feb 2014 & Rusty Russell & { We'll add more non-normative sections with hints for

implementing registers such as PCI class, status

and command registers.
 } \\
\hline
218 & 10 Feb 2014 & Rusty Russell & { example code does not have to be optimal but it

seems cleaner to disable interrupts after we

recheck the ring empty state.
 } \\
\hline
217 & 10 Feb 2014 & Rusty Russell & { patch feedback-7-orig.patch
 } \\
\hline
216 & 10 Feb 2014 & Rusty Russell & { patch feedback-6.patch
 } \\
\hline
215 & 10 Feb 2014 & Rusty Russell & { patch feedback-5.patch
 } \\
\hline
214 & 10 Feb 2014 & Rusty Russell & { patch feedback-4.patch
 } \\
\hline
213 & 10 Feb 2014 & Rusty Russell & { PCI Section Rework

1) Minor changes from must to MUST etc.

2) More references using \textbackslash ref.

3) Move section on capabilities first, before we talk about the common

   config layout.  The previous order made sense for legacy.

4) Make explicit subsections for each type of capability and move more

   information into them.

5) Make it clear that there must be one or more.

6) Include 'struct virtio_pci_cap cap;' in struct virtio_pci_cfg_cap to

   match virtio_pci_notify_cap.

7) Explicitly note there's no way to negotiate the queue size for a

   legacy device.

8) Fix old language on config change event: config is not in the pci

   configuration space.

9) Explicitly state what the driver should do to use virtio_pci_cfg_cap.
 } \\
\hline
212 & 10 Feb 2014 & Rusty Russell & { C struct specifications.

Explicitly specify that our C struct specifications are without padding,

and add some definitions for our integer data types.

[ Rusty - added /* comments */ and removed redundant old le* explanation ]
 } \\
\hline
207 & 07 Feb 2014 & Rusty Russell & { Cleanup and setup clarifications

1) Explicitly allow drivers to read config space during feature

   negotiation.

2) Add the concept of a "live" virtqueue, and explicitly disallow

   moving it backwards or changing descriptors.
 } \\
\hline
204 & 07 Feb 2014 & Rusty Russell & { block: legacy SCSI command fix.

When describing the historical layout requirements, it says

 "status field is a separate read-only buffer of size 1 byte, by itself."

That's clearly wrong, as it says above "The final status byte is written by the device"
 } \\
\hline
203 & 06 Feb 2014 & Rusty Russell & { whitespace: make all examples unindented, and avoid tabs.

This makes the formatting far nicer.  Applying now as it touches almost

all examples and layouts, so we can rebase future changes on top of

common ground.

(Based on feedback from Thomas Huth for one example, and generalized).
 } \\
\hline
201 & 31 Jan 2014 & Rusty Russell & { 3.2.1: Language tightening.

1) Lots of "we", replace with "the driver".

2) Use MAY and MUST NOT for spurious notifications.

3) Don't refer to PCI configuration space for notification.
 } \\
\hline
198 & 29 Jan 2014 & Pawel Moll & { 4.1.2.5: Legacy: PCI Device Layout: fix PCI header fields order

The order of the fields in the legacy PCI header seems to get

messed up in the new spec, with the "Queue Address" moved

behind "Queue Notify". According to the 0.9.5 version of the spec

it should be:

* Device Features 32

* Driver Features 32

* Queue Address 32

* Queue Size 16

* Queue Select 16

* Queue Notify 16

* Device Status 8

* ISR Status 8

--

1.8.3.2
 } \\
\hline
197 & 29 Jan 2014 & Rusty Russell & { Feedback \#3: Feedback from Pranavkumar Sawargaonkar (VIRTIO_CONSOLE_F_EMERG_WRITE)

Document: virtio-v1.0-csprd01

Number: 3

Date: Tue, 21 Jan 2014 15:09:54 +0530
Link to Mail: \url{https://lists.oasis-open.org/archives/virtio-comment/201401/msg00037.html}

Commenter name: Pranavkumar Sawargaonkar <pranavkumar@linaro.org>

Approved at meeting 2014-01-28:
	\url{https://lists.oasis-open.org/archives/virtio/201401/msg00054.html}
 } \\
\hline
196 & 29 Jan 2014 & Rusty Russell & { Feedback \#2: More feedback from  Thomas Huth

Document: virtio-v1.0-csprd01

Number: 2

Date: Fri, 10 Jan 2014 13:49:49 +0100
Link to Mail: \url{https://lists.oasis-open.org/archives/virtio-comment/201401/msg00001.html}

Commenter name: Thomas Huth <thuth@linux.vnet.ibm.com>

Approved at meeting 2014-01-28:
	\url{https://lists.oasis-open.org/archives/virtio/201401/msg00054.html}
 } \\
\hline
195 & 29 Jan 2014 & Rusty Russell & { Feedback \#1: fixes from Thomas Huth

Document: virtio-v1.0-csprd01

Number: 1

Date: Fri, 10 Jan 2014 11:01:44 +0100
Link to Mail: \url{https://lists.oasis-open.org/archives/virtio-comment/201401/msg00000.html}

Commenter name: Thomas Huth <thuth@linux.vnet.ibm.com>

Approved at meeting: 2014-01-28
	\url{https://lists.oasis-open.org/archives/virtio/201401/msg00054.html}
 } \\
\hline
194 & 28 Jan 2014 & Pawel Moll & { mmio: Move QueueReady register from offset 0x03c to 0x044

Legacy devices have QueueAlign register at 0x03c. To stay

on the safe side and avoid any potential clashes (also to

be able to abort any wrong writes), move it to previously

unused offset 0x044.
 } \\
\hline
193 & 23 Jan 2014 & Cornelia Huck & { virtio-ccw: fix set_revision payload definition

The members of struct virtio_rev_info are big endian: use be16 types.
 } \\
\hline
191 & 23 Jan 2014 & Rusty Russell & { Formatting: fix feature bits for console device.

Make them a description list like every other device.
 } \\
\hline
190 & 23 Jan 2014 & Rusty Russell & { Michael's patch adding MQ support added some u16s; they are u16 in

legacy mode but should be le16 for modern devices.
 } \\
\hline
185 & 17 Jan 2014 & Rusty Russell & { net/multiqueue: tighten wording
 } \\
\hline
184 & 17 Jan 2014 & Rusty Russell & { Fixes for first WD front page.

Based on feedback from Paul Knight <paul.knight@oasis-open.org>.
 } \\
\hline
179 & 03 Jan 2014 & Pawel Moll & { mmio: Obviously wrong notification register name

The "4.2.3.3 Notifying The Device" section said "writing

the index of the updated queue to the QueueNum". This

is obviously wrong - should read "QueueNotify".
 } \\
\hline
178 & 16 Dec 2013 & Pawel Moll & { title \& acknowledgements: Make ARM less limited

... by removing the "Limited" bit of the name.
 } \\
\hline
177 & 16 Dec 2013 & Pawel Moll & { 2.3.2 MMIO: Configuration space offset corrected

The offset in the MMIO configuration space description

(table 4.1) became wrong at some time (0x0fff). Fixed.
 } \\
\hline
176 & 12 Dec 2013 & Pawel Moll & { 2.3.2 MMIO: Notifications \& interrupts clarifications

(Hopefully) clarified the way notifications are being

passed between the device and the driver and about

the meaning of the interrupt registers.
 } \\
\hline
175 & 12 Dec 2013 & Pawel Moll & { 1. Introduction: Removed left-over "PCI"

The "Extensible" paragraph of the introduction still

referred to "Virtio PCI devices". Changed to

"Virtio devices".
 } \\
\hline
174 & 12 Dec 2013 & Pawel Moll & { 2.3.2 MMIO: Further clarifications

Clarified driver behaviour for out-of-spec MagicValue,

Version and DeviceID values.
 } \\
\hline
173 & 12 Dec 2013 & Cornelia Huck & { ccw: feature bit endianness

In contrast to the other values transmitted in ccw payload, feature bits

are little endian. Fix it in the structure definition.
 } \\
\hline
172 & 12 Dec 2013 & Cornelia Huck & { ccw: clarify passing of subchannel id

Make clear that the upper half of the register must be ignored, just

like normal I/O instructions do.
 } \\
\hline
171 & 12 Dec 2013 & Cornelia Huck & { ccw: Tighten specification language.

must -> MUST changes, removed inappropriate mays.
 } \\
\hline
170 & 09 Dec 2013 & Pawel Moll & { 2.3.2 MMIO: LaTeXisation

Converter the register layout descriptions into

tables.

Also hardened the specification language, using

MUSTs and MUST NOTs.
 } \\
\hline
168 & 09 Dec 2013 & Michael S. Tsirkin & { commands-pdf.tex: align title page

Section titles are currently misaligned on the title

page. This patch aligns them back.
 } \\
\hline
167 & 09 Dec 2013 & Michael S. Tsirkin & { net: document VIRTIO_NET_F_MAC_ADDR

VIRTIO-50

    virtio-spec: set mac address by a new vq command

Approved Dec 3, 2013
 } \\
\hline
166 & 09 Dec 2013 & Michael S. Tsirkin & { net: add _F_MQ support

VIRTIO-49

Includes git commits:

    virtio-spec: fix two typos

    virtio-spec: virtio network device multiqueue support

    net: add note that you can defer rx queue init until mq enable.

Approved Dec 3, 2013
 } \\
\hline
