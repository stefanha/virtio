478 & 15 Mar 2015 & Cornelia Huck & {VIRTIO-129: legacy:
clean up virtqueue layout definitions

Generalize "Legacy Interfaces: A Note on Virtqueue Layout" to allow
for different alignment requirements. Have pci and ccw refer to that
section for legacy devices. Remove the double definition of virtqueue
alignment (which referred to legacy, but was not tagged as such) from
the ccw section.
See \ref{sec:Basic Facilities of a Virtio Device / Virtqueues /
Legacy Interfaces: A Note on Virtqueue Layout}, \ref{sec:Virtio
Transport Options / Virtio Over PCI Bus / PCI-specific
Initialization And Device Operation / Device Initialization /
Virtqueue Configuration / Legacy Interface: A Note on Virtqueue
Configuration} and \ref{sec:Virtio Transport Options / Virtio
over channel I/O / Device Initialization / Configuring a
Virtqueue / Legacy Interface: A Note on Configuring a Virtqueue}.
 } \\
\hline
479 & 15 Mar 2015 & Cornelia Huck & {VIRTIO-118:
ccw: clarify basic channel commands

"Basic channel commands" seems to be not as clear as it
could, so let's spell out which channel commands we refer to.
See \ref{sec:Virtio Transport Options / Virtio over channel I/O /
Basic Concepts}.
} \\
\hline
479 & 15 Mar 2015 & Cornelia Huck & {VIRTIO-116:
ccw: allow WRITE_STATUS to fail
    
We want to be able to fail setting a status on the device
(e.g.  FEATURES_OK if the device can't work with the features
negotiated).
The easiest way to do that is to allow the device to fail the
WRITE_STATUS command by posting a command reject.
See \ref{sec:Virtio Transport Options / Virtio over channel I/O /
Device Initialization / Communicating Status Information}.
 } \\
\hline
485 & 15 Mar 2015 & Jason Wang & {VIRTIO-135:
virtio-ring: comment fixup
    
virtio_ring.h included with spec has this text:
/* Support for avail_idx and used_idx fields */
it should really refer to avail_event and used_event.
See Appendix \ref{sec:virtio-ring.h}.
 } \\
\hline
486 & 15 Mar 2015 & Jason Wang & {VIRTIO-136:
document idx field in virtqueue used ring

Section \ref{sec:Basic Facilities of a Virtio Device / Virtqueues
/ The Virtqueue Used Ring} The Virtqueue Used Ring
listed the idx field, but never documented it.
See \ref{sec:Basic Facilities of a Virtio Device / Virtqueues /
The Virtqueue Used Ring}.
 } \\
\hline
487 & 15 Mar 2015 & Rusty Russell & {VIRTIO-130:
ISR status: Fix incorrect diagram

ISR status capability diagram has the "Device Configuration
Interrupt " as bit 0, and the "Queue Interrupt" as bit 1. This is
the wrong way around: it disagrees with the legacy
implementations, as well as the spec elsewhere.

All current guests correctly follow the text, fix
up the diagram to match.
See \ref{sec:Virtio Transport Options / Virtio Over PCI Bus / PCI
Device Layout / ISR status capability}.
 } \\
\hline
488 & 15 Mar 2015 & Rusty Russell & {VIRTIO-133:
Change 4.1.5.1.2.1 to device requirement

4.1.5.1.2.1 is incorrectly labelled as a driver requirement; it's
self-evidently referring to the device.
See \ref{sec:Conformance / Driver Conformance / PCI Driver
Conformance}, \ref{sec:Conformance / Device Conformance / PCI
Device Conformance} and \ref{devicenormative:Virtio
Transport Options / Virtio Over PCI Bus / PCI-specific
Initialization And Device Operation / Device Initialization /
Non-transitional Device With Legacy Driver}.
 } \\
\hline
504 & 22 Apr 2015 & Rusty Russell & {VIRTIO-137:
define the meaning and requirements of the len field.
    
We said what it was for, and noted why.  We didn't place any
requirements on it, nor clearly spell out the implications of its use.
    
This clarification comes particularly from noticing that QEMU
didn't set len correctly, and philosophising over the correct value
when an error has occurred.
See \ref{sec:Basic Facilities of a Virtio Device / Virtqueues /
The Virtqueue Used Ring}, \ref{devicenormative:Basic Facilities
of a Virtio Device / Virtqueues / The Virtqueue Used Ring} and
\ref{sec:Basic Facilities of a Virtio Device / Virtqueues / The
Virtqueue Used Ring}.
 } \\
\hline
506 & 22 Apr 2015 & Michael S. Tsirkin & {VIRTIO-138:
multiple errors: Non-transitional With Legacy

virtio 1.0 has two sections titled "Non-transitional Device With
Legacy Driver" the first says devices SHOULD fail, the second
says devices MUST fail.  Clearly a mistake.

Other issues: devices don't really fail - they cause drivers to
fail. second section seems to be in the wrong place, and also
have a section followed by subsection with no explanatory text in
between, which is ugly.
Finally, this text was originally ritten to handle buggy windows
drivers gracefully, but later we changed device IDs so it's not
really required there. Might be handy for some other buggy legacy
drivers, though no such drivers are known.

To fix, drop the duplicate section variant, add some explanatory
text, clarify what does "same ID" mean here, and clarify
that the work-around is only needed if a buggy driver
is known to bind to a transitional device.

See \ref{sec:Virtio Transport Options / Virtio
Over PCI Bus / PCI Device Layout / Non-transitional Device With
Legacy Driver: A Note on PCI Device Layout},
\ref{devicenormative:Virtio Transport Options / Virtio Over PCI
Bus / PCI-specific Initialization And Device Operation / Device
Initialization / Non-transitional Device With Legacy Driver} and
\ref{sec:Virtio Transport Options / Virtio Over PCI Bus /
PCI-specific Initialization And Device Operation / Device
Initialization}.
} \\
\hline
508 & 22 Apr 2015 & Michael S. Tsirkin & {VIRTIO-139:
pci: missing documentation for dealing with 64 bit config fields
    
pci spec says what width access to use for 32, 16 and 8
bit fields, but does not explicitly say what to do for
32 bit fields. As we have text that says driver must
treat 64 bit accesses as non-atomic, this seems
to imply driver should always do two 32 bit wide accesses.

Let's make this an explicit requirement, and require
devices to support this.

See \ref{sec:Virtio Transport Options / Virtio Over PCI Bus / PCI
Device Layout}, \ref{drivernormative:Virtio Transport Options /
Virtio Over PCI Bus / PCI Device Layout},
\ref{devicenormative:Virtio Transport Options / Virtio Over PCI
Bus / PCI Device Layout} and \ref{sec:Conformance / Driver
Conformance / PCI Driver Conformance}.
 } \\
\hline
509 & 22 Apr 2015 & Michael S. Tsirkin & {balloon:
MUST -> has to

MUST shouldn't be used outside normative statements,
that's confusing. Replace with "has to".

See \ref{sec:Device Types / Memory Balloon Device / Feature
bits}.
 } \\
\hline
510 & 22 Apr 2015 & Michael S. Tsirkin & {conformance:
add VIRTIO-137 statement links

Add links to new conformance statements added to
resolve VIRTIO-137 (describing used ring entry len usage).

See \ref{sec:Conformance / Device Conformance}
and \ref{sec:Conformance / Driver Conformance}.
 } \\
\hline
517 & 22 Apr 2015 & Michael S. Tsirkin & {acknowledgements:
contributors+minor fixup

acknowledge feedback by Jason Wang, add Richard Sohn who
joined the TC, sort acknowledged reviewers alphabetically.

See \ref{chap:Acknowledgements}.
} \\
\hline
520 & 30 Apr 2015 & James Bottomley & {VIRTIO-140:
give explicit guidance on the use of 64 bit fields

Just saying 64 bit fields may not be atomic is true, but less
helpful than it might be.  Add explicit guidance about what the
consequences of non-atomicity are.

See \ref{sec:Creating New Device Types / What Device
Configuration Space Layout?}
} \\
\hline
521 & 30 Apr 2015 & Rusty Russell & {VIRTIO-134:
Spell out details of indirect elements in chains
    
1) It's implied that a chain terminates with an indirect descriptor (since
VIRTIO-15) but we didn't spell out that a device MUST NOT
continue it.
    
2) We allow [direct]->[direct]->[indirect], and qemu and
bhyve both accept it.  Make it clear that this is valid, thus devices MUST
handle it.

See \ref{drivernormative:Basic Facilities of a Virtio Device /
Virtqueues / The Virtqueue Descriptor Table / Indirect
Descriptors} and \ref{devicenormative:Basic Facilities of a
Virtio Device / Virtqueues / The Virtqueue Descriptor Table /
Indirect Descriptors}
} \\
\hline
522 & 30 Apr 2015 & Michael S. Tsirkin & {VIRTIO-141:
used ring: specify legacy behaviour for len field

many hypervisors implemented len field incorrectly.
Document existing bugs in the legacy sections.

See \ref{sec:Basic Facilities of a Virtio Device / Virtqueues
/ The Virtqueue Used Ring/ Legacy Interface: The Virtqueue Used
Ring}, \ref{sec:Device Types / Network Device / Device Operation
/ Legacy Interface: Device Operation}, \ref{sec:Device Types /
Block Device / Device Operation / Legacy Interface: Device
Operation}, \ref{sec:Device Types / Console Device / Device
Operation / Legacy Interface: Device Operation}, \ref{sec:Device
Types / Memory Balloon Device / Device Operation / Legacy
Interface: Device Operation}, \ref{sec:Device
Types / SCSI Host Device / Device Operation / Legacy
Interface: Device Operation} and \ref{sec:Conformance / Legacy
Interface: Transitional Device and Transitional Driver
Conformance}.
} \\
\hline
523 & 30 Apr 2015 & Michael S. Tsirkin & {VIRTIO-142:
entropy device: typo fix

Current text: "The driver MUST examine the length written by the
driver" makes no sense. length is written by the device.

See \ref{drivernormative:Device Types / Entropy Device / Device
Operation}.
} \\
\hline
526 & 18 May 2015 & Michael S. Tsirkin & {VIRTIO-143:
balloon: transitional device support

Support a transitional balloon device: this has the advantage of supporting
existing drivers, transparently, as well as transports that don't allow mixing
virtio 0 and virtio 1 devices. And balloon is an easy device to test, so it's
also useful for people to test virtio core handling of transitional devices.

Three issues with legacy hypervisors have been identified:
\begin{enumerate}
\item
Actual value is actually used, and is necessary for management
to work. Luckily 4 byte config space writes are now atomic.
When using old guests, hypervisors can detect access to the last byte.
When using old hypervisors, drivers can use atomic 4-byte accesses.
\item Hypervisors actually didn't ignore the stats from the first
buffer supplied. This means the values there would be
incorrect until hypervisor resends the request.
Add a note suggesting hypervisors ignore the 1st buffer.
\item QEMU simply over-writes stats from each buffer it gets.
Thus if driver supplies a different subset of stats
on each request, stale values will be there.
Require drivers to supply the same subset on each
request. This also gives us a simple way to figure out
which stats are supported.
\end{enumerate}

See
\ref{sec:Device Types / Memory Balloon Device},
\ref{devicenormative:Virtio Transport Options / Virtio Over PCI Bus / PCI Device Discovery},
\ref{sec:Conformance / Driver Conformance / Traditional Memory Balloon Driver Conformance},
\ref{sec:Conformance / Device Conformance / Traditional Memory Balloon Device Conformance},
\ref{sec:Conformance / Legacy Interface: Transitional Device and Transitional Driver Conformance},
\ref{sec:Conformance / Device Conformance} and \ref{sec:Conformance / Driver Conformance}.
} \\
\hline
527 & 18 May 2015 & Michael S. Tsirkin & {VIRTIO-126:
document deflate on oom

Document the new option, and also clarify behaviour
without it.

In particular, actual field is not the
actual number of pages in the balloon as
driver might do inflate followed by deflate.

Also, device isn't always driven by interrupts,
driver can inflate/deflate in response to e.g.
memory compaction.

See \ref{sec:Device Types / Memory Balloon Device / Feature bits},
\ref{sec:Device Types / Memory Balloon Device / Device Operation} and
\ref{drivernormative:Device Types / Memory Balloon Device / Device Operation}.
} \\
\hline
528 & 18 May 2015 & Michael S. Tsirkin & {VIRTIO-123:
network device: xmit/receive cleanup

Fix up multiple issues in xmit/receive sections:
\begin{itemize}
   \item drop MAY/MUST/SHOULD outside normative statements
   \item spell out conformance requirements for both drivers and
      devices, for xmit and receive paths
   \item document the missing VIRTIO_NET_HDR_F_DATA_VALID
   \item document handling of unrecognized flag bits so we can extend
      flags in the future, similar to VIRTIO_NET_HDR_F_DATA_VALID
\end{itemize}

\ref{sec:Device Types / Network Device / Device Initialization},
\ref{drivernormative:Device Types / Network Device / Device Operation / Packet Transmission},
\ref{devicenormative:Device Types / Network Device / Device Operation / Packet Transmission},
\ref{sec:Device Types / Network Device / Device Operation / Processing of Incoming Packets},
\ref{sec:Conformance / Driver Conformance / Network Driver Conformance} and
\ref{sec:Conformance / Device Conformance / Network Device Conformance}.
} \\
\hline
529 & 18 May 2015 & Michael S. Tsirkin & {VIRTIO-124:
network device: document VIRTIO_NET_F_CTRL_RX_EXTRA

See
\ref{sec:Device Types / Network Device / Device Operation / Control Virtqueue / Packet Receive Filtering},
\ref{sec:Device Types / Network Device / Device Operation / Control Virtqueue / Setting MAC Address Filtering},
\ref{sec:Conformance / Driver Conformance / Network Driver Conformance} and
\ref{sec:Conformance / Device Conformance / Network Device Conformance}.
} \\
\hline
